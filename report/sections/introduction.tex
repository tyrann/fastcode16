Cryptography is a complex and widespread computational problem in modern computer science research and applications. To ensure the confidentiality of the information exchange, encryption provides the core foundation of a secure information system. Among different cryptographic problems, key exchange is an essential topic for two distant users to be able to share a secret without meeting physically. Internet technology companies are interested in efficient yet high performance algorithms to fulfill the needs of exchanging keys in this age where network security has been concerned more than ever. Various key exchange protocols have been generated.(TODO: add ecdh) Elliptic Curve Cryptography (ECC) has shown its advantages over traditional RSA (Rivest, Shamir and Adleman). By using less key sizes, ECC can apply a more efficient method without reducing the secure level of the encryption, whereas RSA requires the involvement of larger primer number to reach the same goal \cite{Malik:2010}. 

As an important application domain of ECC, Elliptic Curve Diffie-Hellman (ECDH) key exchange has attracted researcher's attention due to its safety and efficiency. A fast implementation of ECDH is highly in demand, as the number of Internet users increases drastically with the growing needs for information security and the speed of computation. ECDH is widely used and there are some potential to be improved in its computational performance among the existed implementations. The challenge lies in the complex computational models and its natural requirements for large integers. In this paper, we propose a fast implementation of high performance ECDH key exchange.By comparing with popular cryptography library we proved the feasibility of our improvement of our optimization methods.  (TODO: OpenSSL)

Brown provides the backbone of this project by providing a general standard of Elliptic Curve Cryptography\cite{Brown:2009} and specifying a well-established deployment requirements. Malik's work regarding a particular implementation in a FPGA card \cite{Malik:2010} shows the possibility of performance oriented optimization. The use of projective coordinates, as the most efficient optimization methods during our implementation, was proven by Blake et al.\cite{Blake:1999} as an effective method for expensive field inversions. In this paper we combine multiply approaches of optimizations and achieved satisfactory performance improvement and execution speedup.  

We give a thorough description of background in section 2, proposed optimization methods in section 3 and show our experimental results in section 4. In Section 5, we discuss and analyze our approach.