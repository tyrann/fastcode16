Cryptography is a complex and widespread computational problem in modern computer science research and applications. To ensure the confidentiality of the information exchange, encryption provides the core foundation of a secure information system. Among different cryptographic problems, key exchange is an essential topic for two distant users to be able to agree on a shared secret without meeting physically. Internet technology companies are interested in high performance implementations to fulfill the needs of exchanging keys nowadays. Elliptic Curve Cryptography (ECC) has shown its advantages over traditional RSA (Rivest, Shamir and Adleman). By using smaller key sizes, ECC can provide a more efficient method without reducing the security level of the encryption, whereas RSA requires the involvement of larger prime numbers to reach the same goal\cite{Malik:2010}.

As an important application domain of ECC, Elliptic Curve Diffie-Hellman (ECDH) key exchange has attracted attention due to its security and efficiency. ECDH is widely used and there is room for improvement in its computational performance among the existing implementations. The challenge lies in complex computational models and multi-preci\-sion integers. The propagation of the carry flag causes unavoidable dependencies that make instruction-level parallelism difficult. In this paper we propose a fast implementation of high performance ECDH key exchange. We combined multiple approaches of optimizations achieving satisfactory performance and execution speed.

Brown\cite{Brown:2009}  provides the theoretical backbone of this project by giving a general standard of Elliptic Curve Cryptography and specifying well-established deployment requirements\cite{Brown:2010}. Malik's work regarding a particular implementation in a FPGA card\cite{Malik:2010} shows the possibility of performance oriented optimization. OpenSSL is a wide spread open source cryptography software which provides a state-of-the-art implementation for ECDH\cite{Emilia:2011}. In the end we test our optimization result by comparing with OpenSSL.