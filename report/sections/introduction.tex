Cryptography is a complex and widespread computational problem in modern computer science research and applications. To ensure the confidentiality of the information exchange, encryption provides the core foundation of a secure information system. Among different cryptographic problems, key exchange is an essential topic for two distant users to be able to share a secret without meeting physically. Internet technology companies are interested in efficient yet high performance implementations to fulfill the needs of exchanging keys nowadays. Elliptic Curve Cryptography (ECC) has shown its advantages over traditional RSA (Rivest, Shamir and Adleman). By using less key sizes, ECC can apply a more efficient method without reducing the secure level of the encryption, whereas RSA requires the involvement of larger primer number to reach the same goal \cite{Malik:2010}.

As an important application domain of ECC, Elliptic Curve Diffie-Hellman (ECDH) key exchange has attracted researcher's attention due to its safety and efficiency. Diffie-Hellman scheme is a key agreement scheme that can provide implicit key authentication\cite{Brown:2009} and exchange shared secrecy. ECDH is widely used and there are some potential to be improved in its computational performance among the existed implementations. The challenge lies in the complex computational models and multi-precision integers when the propagation of the carry causes unavoidable dependency. In this paper, we propose a fast implementation of high performance ECDH key exchange. In this paper we combine multiple approaches of optimizations and achieved satisfactory performance improvement and execution speedup.

Brown provides the backbone of this project by giving a general standard of Elliptic Curve Cryptography\cite{Brown:2009} and specifying a well-established deployment requirements\cite{Brown:2010}. Malik's work regarding a particular implementation in a FPGA card \cite{Malik:2010} shows the possibility of performance oriented optimization. The use of projective coordinates was proven by Blake et al.\cite{Blake:1999} as an effective method for expensive field inversions. OpenSSL is a world wide open source cryptography software which can fulfill the functionality for ECDH\cite{Emilia:2011}. In the end we test our optimization result by comparing with OpenSSL.