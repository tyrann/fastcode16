The purpose of this section is to familiarize the reader with the mathematical foundations of the project. It is heavily based on \cite{Brown:2009}, \cite{Washington:2008}.

\mypar{Finite Field arithmetic} For elliptic curve cryptography one is mainly interested in the prime finite field $\mathbb{F}_p$ and the characteristic 2 finite field $\mathbb{F}_{2^m}$. In this paper only elliptic curves over $\mathbb{F}_p$ are discussed.
 
\mypar{Elliptic Curve over $\mathbb{F}_p$ }
An elliptic curve is defined by the equation
\begin{equation}\label{eq:defining_eq_ec}
y^2 \equiv x^3 + ax + b \quad (\text{mod } p)
\end{equation}
where $p$ is an odd prime number and $a,b \in \mathbb{F}_p$ are the parameters of the curve. Furthermore $a, b$ need to satisfy $4a^3 + 27b^2 \not\equiv 0$. The elliptic curve $E\left(\mathbb{F}_p\right)$ consists of the Points $P=(x,y)$ $x,y \in \mathbb{F}_p$ satisfying (\ref{eq:defining_eq_ec}). Additionally we introduce $\mathcal{O}$ the so called point at infinity. As soon as the addition is defined, this will be the neutral element of the group. How the addition can be defined see \cite{Brown:2009}.

\mypar{Diffie Hellman key exchange}
Alice and Bob want to establish a secret over a public channel. We assume that the elliptic curve parameter: a prime $p$, $a, b \in \mathbb{F}_p$ a point $G$ with high order are publicly known.
\begin{enumerate}
\item{Alice chooses a secret integer $u$, computes $G_u = uG$, and sends $G_u$ to Bob.}
\item{Bob chooses a secret integer $v$, computes $G_v = vG$, and sends $G_v$ to Alice.}
\item{Alice computes $uG_v = uvG$}
\item{Bob computes $vG_u = vuG$.}
\end{enumerate}
One can verify that $uG_v = uvG = vuG = v G_u$. An eavesdropper knows $G, G_u, G_v$ and his goal is to calculate $uvG$. This is known as the Diffie-Hellman Problem and is assumed to be a hard problem.

\mypar{Double-and-add method}
In order to implement the Diffie-Hellman key exchange we need to calculate $dP$ fast.
Input $P \in E(\mathbb{F}_p)$, $d \in \mathbb{N}$\\
Output: $d\cdot P \in E(\mathbb{F}_p)$
\begin{lstlisting}[frame=single, mathescape=true, captionpos=b, caption=double-and-add method]
$N$ <- $P$
$Q$ <- $\mathcal{O}$
for i from 0 to m do
  if $d_i$ = 1 then
    $Q$ <- point_add($Q$, $N$)
  $N$ <- point_double($N$)
return $Q$
\end{lstlisting}
where $d = d_0 + d_1 2 + ... + d_m 2^m \quad d_i \, \in \{0,1\}$

\mypar{Complexity}

\mypar{Cost measure}