The idea of this section is to give an overview about our implementation and the optimizations we did. We distinguish between four versions of our code. \textbf{Baseline} is the first naive implementation. \textbf{Memory optimization} uses the same algorithm as the baseline, but the memory management is improved. The version \textbf{Jacobian coordinates} can be viewed as a second baseline. It includes a lot of algorithmic improvements, which we decided to add after analyzing the OpenSSL implementation. \textbf{Final} marks our latest version.

\mypar{Baseline} 
Since ECDH requires multi-precision integer arithmetic, 64-bit integer operations are not sufficient. We therefore had to find a representation for arbitrary sized integers. Our choice is shown in Listing \ref{lst:bigint}. We decided not to use a library, in order to have full flexibility in the optimizations. As a next step we implemented shifting, modular addition, subtraction, multiplication and division. Furthermore we implemented addition of points (affine coordinates) and multiplication of a point with a scalar (double-and-add method). All of the functions were heavily tested with unit tests.

\begin{lstlisting}[frame=single,  captionpos=b, caption=representation of the arbitrary size integers, label=lst:bigint, language=c]
typedef struct 
{
    uint64 significant_blocks;    
    uint64 blocks[BI_BLOCKS_COUNT]; 
} __BigInt;
\end{lstlisting}

\mypar{Memory optimization}

A first analysis of our naive implementation revealed that the creation and destruction of \texttt{BigInt} objects caused millions of \texttt{malloc}/\texttt{free} operations to be executed for each key computation. This results in poor locality and lot of time spent in operating system calls.

Our solution consists of a single contiguous buffer with enough space for all the required \texttt{BigInt} objects. The buffer is allocated at initialization time and each function can access the single objects using tags like in the example in Listing \ref{lst:access}:

\begin{lstlisting}[frame=single, mathescape=true, captionpos=b, label=lst:access, caption=Access by tag example]
BigInt shared_v = GET_BIGINT_PTR(
    BI_ECDH_SHAREDV_TAG);
\end{lstlisting}

For the scope of our project we assume that the maximum size of the prime $p$ is $521$ bits. We conclude that each \texttt{BigInt} object fits in at most $192$ bytes. Since there are $134$ distinct objects we get a total of $25.8KB$, that fit in the $32KB$ L1 data cache of our target system.

With this approach we have optimal spatial and temporal locality, minimizing the number of cache misses to the compulsory ones. Moreover we only have one \texttt{malloc} operation. Since the position of each object in the buffer is known at compile time, all memory loads can be performed with simple offsetting from a single common base address.


Furthermore, the \texttt{blocks} field of a \texttt{BigInt} object is 32 bytes aligned. This allows the C library to use optimized SIMD-based implementations of \texttt{memcpy} and \texttt{memset}.

\mypar{Jacobian coordinates}
In this code version a lot of algorithmic optimizations were performed. The number of Montgomery conversions was reduced significantly since the baseline implementation was suboptimal. Furthermore the base $b$ for the Montgomery multiplication was changed from 2 to $2^{64}$.

When considering the Diffie Hellman key exchange, one can see that $u \cdot G$ and $v \cdot G$ have to be calculated, where $G$ is the base point, which is previously known. This allows to precompute the values $2 \cdot G$, $4 \cdot G$, $8 \cdot G$,..., which otherwise have to be calculated on line 9 in Listing \ref{lst:double_and_add}.

Changing from affine coordinates to Jacobian coordinates gave another significant speedup. In affine coordinates the modular division and the multiplication were the bottlenecks. Introducing Jacobian coordinates allowed us to focus on optimizing the multiplication, since the division is only needed for converting from Jacobian to affine coordinates.

\mypar{Final}
The following paragraphs describe the last round of optimizations that we introduced. Here we concentrate mainly on performance. 

1) \emph{Intel ADX and BMI2 \cite{Otzturk:2012}}. 
The handling of the carry flag is a well-known problem on the x86 architecture. Developers usually have to use inline assembly and the \texttt{ADC} instruction or perform additions on smaller words, using the higher bits as carry flag like in our naive implementation.

To overcome this problem Intel introduced the ADX extension, consisting of the \texttt{ADCX} (addition with carry flag) and \texttt{ADOX} (addition with overflow flag) instructions. ADX instructions redefine the use of the FLAGS register, using CF and OF respectively as carry flags and leaving the other flags untouched. This allows to perform two parallel chains of additions without conflicts in the carry flags.

These instructions are accessible to C/C++ developers through the \texttt{\_addcarryx\_u64} intrinsic. Moreover Intel defines the \texttt{\_subborrow\_u64} intrinsic, that performs $64$bit subtractions with borrow flag.

The BMI2 extension introduces the \texttt{MULX} instruction that multiplies two $64$bit integers, saving the $128$bit result in two $64$bit registers. This instruction doesn't affect the arithmetic flags, thus it can be safely mixed with the ADX instructions. The relative intrinsic is \texttt{\_mulx\_u64}. 

Our optimized code makes heavy use of these intrinsics, that significantly improve the speed of our application. However the latest versions of GCC and ICC don't support ADX as expected. The assembly output shows that they emit simple \texttt{ADC} instructions, adding the overhead of saving and loading the carry flag in presence of multiple carry chains.

2) \emph{Montgomery multiplication}.
Running our application with callgrind after introducing Jacobian coordinates, we noticed that the bottleneck of our algorithm is the multiplication in Montgomery space. This covers more than $80\%$ of the execution time, making optimization of this step crucial.
 
The core of the multiplication is the following operation:
$$res = \left(res + y * x_i + p * u\right) \gg 64$$
where $x_i$ and $u$ are $64$bit numbers, the remaining variables are BigInts. The operation was optimized in different steps:
\begin{enumerate}
\item The basic BigInt operations were stitched into a single function, reducing the number of accesses to the cache and thus the operational intensity.
\item Since our BigInt representation is based on $64$bit blocks, shifting a number by $64$bits is equivalent to shifting by one block. Here the right shift operation has been made implicit by writing the resulting blocks at index $i-1$ instead of $i$.
\item Since the length of the loop is determined by the size of $p$ we used macros for redefining the function with different fixed loop lengths. By doing so, the compiler can easily unroll the loops, using constant offsets for array accesses. This drastically reduces the number of index operations. 
\item We observed that the compiler is not able to emit \texttt{ADCX} and \texttt{ADOX}, thus we decided to write the instructions manually as inline assembly. 
\end{enumerate}

\mypar{ADX vs AVX2}  
In the literature we found no reference to successful use of SIMD (Single Instruction Multiple Data) for ECC. We verified the unsuitability of SIMD for this application by computing analytical lower bounds for our ADX implementation and for two possible AVX2 approaches.
We estimate the number of cycles required by one iteration when processing $4$ parallel carry chains, i.e. $2$ montgomery multiplications. Results are summarized in table \ref{tbl-adx-avx2}.

\begin{table}
\centering
\caption{Lower bounds for different approaches}
\label{tbl-adx-avx2}
\begin{tabular}{ l l l}
	\hline
 	Approach & Cycles/iteration & Bottleneck \\ \hline
 	ADX & 8 & ADX througput \\
  	AVX2 32bit & 8 & Carry emulation \\
 	AVX2 64bit & 16 & Carry emulation \\ \hline
\end{tabular}
\end{table}

The main limitation is that AVX2 doesn't have support for carry propagation, thus it must be emulated. In the $32$bit case this can be done by adding $32$bit integers in $64$bit words, using the higher bits as carry. The $64$bit approach requires unsigned integer comparisons, which are expensive on AVX2. Moreover there are additional downsides:
\begin{itemize}
\item Data need to be transposed in the registers after load and before writing 
\item Multiplications have higher latency.
\item Requires two Montgomery multiplications to be independent, which is not always possible.
\end{itemize}
